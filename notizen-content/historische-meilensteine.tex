\section{Historische Meilensteine}
\label{sec:historische-meilensteine}

\begin{itemize}
  \item 1957: Sputnik, Messung der Signale, Erkennen des Doppler-Effektes, Rückrechnen auf Laufzeit möglich
  \item 1964: US-Marine, Navigation mit Radio-Signalen, TRANSIT System
  \begin{enumerate}
      \item 15min von Horizont zu Horizont
      \item nur 6 Sateliten, $\SI{90}{\minute}$ bis zur neuen Position
      \item Dopplerverschiebung messen
      \item von Bill Guier und George Weiffenbach, Dopplermessung für Sputnik
      \item Signale eines Satellitens an bekannter Position für die eigene Positionsbestimmung verwenden
  \end{enumerate}
  \item Kalter Krieg
  \begin{enumerate}
      \item Düsenjäger, schneller Flugzeuge und ballistische Raketen brauchen eine exakte und schnelle Position
      \item Entscheidung zugunsten Satelliten, obwohl es Boden-Radio-Stationen hätten sein können
      \item Entwicklung verschiedener Systeme, einige waren reine Testprojekte, andere halb orbital halb bodengebunden
  \end{enumerate}
   \item Atomuhren für eine genaue Zeitangabe
   \item 1973: Verteidigungsministerium vereinigt die verschiedenen Systeme zum \\\textit{NAVSTAR}-Projekt der Luftwaffe (Joint Programm Office, heute \textit{GPS}),\\Eigenschaften des Systems:
   \begin{enumerate}
       \item Atomuhren in den Satelliten für eine synchronierte Zeit
       \item Militärscher Zweck an 1. Stelle, z.B. Schiffspositionen
       \item Vision der Entwickler ist eine gemeinsame Nutzung von Militär und Zivil
   \end{enumerate}
   \item 1974: Rockwell International(heute Teil von Boeing) bekommt den Auftrag für die ersten Satelliten
   \item 1986: 18 Satelliten sind eingerichtet, das System ist nutzbar
   \item Anfang 1995: alle 24 Satelliten der ersten Generation im Orbit
   \item Anfang der 90er: Neue Generation an Satelliten von Lockhead Martin
   \item 2010: mehr als 30 Satelliten im Betrieb
\end{itemize}
