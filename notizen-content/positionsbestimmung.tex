\section{Positionsbestimmung}
\label{sec:positionsbestimmung}
\subsection{Korrekturen}
\label{sec:korrekturen}

\subsubsection{Phasenverschiebung durch Doppler}
\begin{align}
    φ^\text{S}(t) &= \text{Phase des empfangenen Signals} \\
    φ_\text{E}(t) &= \text{Phase eines internen Referenzsignals}
    \intertext{Beide Signale starten bei}
    t_0 &= 0\;,
    \intertext{die beiden Phasen können dann mit}
    φ^\text{S}(t) &= f\cdot t-\frac{ρ}{\symup{c}}f+f\cdotδ^\text{S}\\
    φ_\text{E}(t) &= f\cdot t+f\cdotδ^\text{E}
    \intertext{beschrieben werden. Die relative Differenz der beiden Phasen ist}
    \frac{\symup{d}f}{f} &= \num{e-12}\;.
    \intertext{Bei einer Trägerfrequenz von}
    f &= \SI{1.5}{\giga\hertz}
    \intertext{folgt eine Ungenauigkeit von}
    \symup{d}f &= \SI{1.5e-3}{\hertz}\;.
    \intertext{Die absolute Differenz}
    φ^\text{S}_\text{E} &= -\frac{ρ}{\symup{c}}f-f\cdot\incrementδ
    \intertext{kann mit der Anzahl $N$ an empfangenen Signalen als}
    φ^\text{S}_\text{E} &= \incrementφ^\text{S}_\text{E}|^t_{t_0}+N
\end{align}
geschrieben werden.
Abbildung 6.1 aus {\small[H-W,L,C], S.90} zeigt dieses.
Die dicke Linie ist der wahre Orbit und die dünne Linie die ohne Phasenverschiebung errechnete.

\subsubsection{Zeitverzögerungen am Signal}
\label{sec:ZeitImSignal}
Erste Definitionen:
\begin{align}
    \text{S}   &= \text{Satellit}\\
    \text{E}   &= \text{Empfänger}\\
    t^\text{S} &= \text{Emissionszeit am Satelliten}\\
    δ^\text{S} &= \text{Verzögerungen im Satelliten, z.B. durch Rechenzeiten}\\
    t^\text{S} &= t^\text{S}(\text{GPS}) + δ^\text{S}\\
    t_\text{E} &= \text{Empfängerzeit}\\
    δ_\text{E} &= \text{Verzögerungen im Empfänger}\\
    t^\text{E} &= t^\text{E}(\text{GPS}) + δ^\text{E}
    \intertext{Damit folgt der Zeitunterschied}
    \increment t &= t_\text{E} - t^\text{S} = \increment t(\text{GPS}) +\incrementδ
    \intertext{und die Pseudoentfernung}
    R &= \symup{c}\increment t = ρ + \symup{c}\incrementδ\;.
    \intertext{Diese kann in eine Taylorreihe entwickelt werden}
    ρ\left(t_\text{E},\;t^\text{S}\right) &= ρ\left(t_\text{E},\;t^\text{S}+\increment t\right) = ρ\left(t^\text{S}\right)+\dot{ρ}\left(t^\text{S}\right)\cdot\increment t\;,
    \intertext{mit der radialen Geschwindigkeit  der Satelliten}
    \dot{ρ} &\approx \SI{0.9}{\kilo\meter\per\second}.
    \intertext{Die Laufzeit des Signal kann mit}
    \increment t &\approx\SI{0.07}{\second}
    \intertext{abgeschätzt werden, das führt zu einem Korrekturterm von}
    \dot{ρ}\left(t^\text{S}\right)\cdot\increment t &\approx\SI{60}{\meter}\;.
\end{align}

% \subsubsection{Relativistische Effekte}
% Von der Verscheibung der Frequenz
% \begin{align}
%     f &= f'\sqrt{1-\frac{v^2}{\symup{c}^2}}\\
%     \frac{f_0'-f_0}{f_0} &= \num{-4.464e-10}
% \end{align}
% ist vor allem die Uhr betroffen.


\subsection{Ortskoordinaten}
Wir folgen dem Unterkapitel \ref{sec:ZeitImSignal}.
\begin{align}
    R_\text{E}^\text{S}(t) &= ρ_\text{E}^\text{S}(t) + \symup{c}\incrementδ_\text{E}^\text{S}(t) \label{eqn:entfernung}\\
    ρ_\text{E}^\text{S}(t) &= \sqrt{\left(X^\text{S}(t)-X_\text{E}\right)^2+\left(Y^\text{S}(t)-Y_\text{E}\right)^2+\left(Z^\text{S}(t)-Z_\text{E}\right)^2}
    \intertext{Mit einem Satelliten kommt man so auf vier Unbekannte, die drei Koordinaten des Empfängers und die Uhrenfehler $δ$. Dieser wird wie folgt angenähert}
    δ(t) &= a_0 + a_1(t-t_\text{r})+ a_2(t-t_\text{r})^2\;,
    \intertext{mit einer Referenzzeit $t_\text{r}$. Damit kann die Gleichung \eqref{eqn:entfernung} so umgestellt werden, dass auf der linken Seite Messgrößen stehen und auf der rechten Seite die Unbekannten}
    R_\text{E}^\text{S}(t)& + \symup{c}δ^\text{S}(t) = ρ_\text{E}^\text{S}(t) + \symup{c}δ_\text{E}\;.
\end{align}
Zur Bestimmung der Position werden vier Satelliten benötigt, wenn die Höhe nicht interessiert, reichen drei Satelliten.

\subsection{Atmosphärische Effekte}
Signale sind Wellenpakete, mit leicht unterschiedlichen Frequenzen. Die Phasen- und Gruppengeschwindigkeit sind gegeben als
\begin{align}
    v_\text{ph} &= \lambda\cdot f\\
    v_\text{gr} &= v_\text{ph}-\lambda\frac{\symup{d}v_\text{ph}}{\symup{d}\lambda}\;,
    \intertext{Die Geschwindigkeit in einem Medium ist mit dem Brechungsindex $n$ gegeben als}
    v &= \frac{\symup{c}}{n}\;.
    \intertext{Mit Reihenentwicklung und Umformungen kommt man zu}
    n_\text{ph} &= 1+\frac{c_2}{f^2}+\frac{c_3}{f^3}+\frac{c_4}{f^4}+\ldots\\
    n_\text{gr} &= 1-\frac{c_2}{f^2}+\ldots
    \intertext{Die Konstanten $c_i$ sind hierbei abhängig von der Elektronendichte. $c_2 = \num{-40.3}N_e$. Pseuentfernung zu lang und Trägerfrequenz beschleungigt. Mit dem Fermat'schen Prinzip}
    s &= \int n\,\symup{d}\,s
\end{align}

\subsubsection{Ionosphäre}
$\SI{50}{\kilo\meter}-\SI{1000}{\kilo\meter}$
Für die Ionosphäre kann das Fermat'sche Prinzip umgeschrieben werden um den Wegunterschied zu bestimmen
\begin{align}
    \increment^\text{Iono} &= \int n\,\symup{d}\,s-\int\symup{d}\,s_0\;.
    \intertext{Mit der totalen Anzahl an Elektronen zwischen Satellit und Empfänger TEC ergibt sich}
    \increment^\text{Iono}_{gr} &= \frac{\num{40.3}}{f^2}\cdot\text{TEC}\;,
    \intertext{negativ für die Phasengeschwindigkeit. Für $\num{e16}$ Elektronen ergibt sich}
    \increment^\text{Iono}_{gr} &= \SI{0.16}{\meter}\;.
\end{align}
Dieser Fehler ist abhängig vom Sonnenstand, der Strecke durch die Ionosphäre und der Aktivität von Sonnenflecken.

\subsubsection{Troposphäre}
Ergänze
\begin{align}
    \increment^\text{Trop} &= \int(n-1)\symup{d}\,s
    \shortintertext{mit}
    N^\text{Trop} &= \num{e16}(n-1)\symup{d}\,s
    \shortintertext{zu}
    \increment^\text{Trop} &= \num{e-16}\int N^\text{Trop}\symup{d}\,s\;.
    \intertext{$N^\text{Trop}$ kann als Summe von Elektronen der trockenen Atmosphäre und der Elektronen aus dem Wasserdampf geschrieben werden. Die Aufteilung ist $\SI{90}{\percent}$ trocken und $\SI{90}{\percent}$ nass. Eine Abschätzung des Fehlers ist auf Meeresoberfläche, im Zenit}
    \increment^\text{Trop} &\approx \SI{2.3}{\meter}
\end{align}

\subsection{Relativistische Effekte}
$S'$ ist das bewegte System
\subsubsection{Doppler Effekt}
Zeitdilatation
\begin{align}
    \gamma &= \frac{1}{\sqrt{1-\frac{v^2}{\symup{c}^2}}} \\
    t &= \gamma\cdot\left(t'+\frac{v}{\symup{c}^2}x'\right)\\
    \increment t &= \gamma\cdot\increment t'
\end{align}
2. Ordnung
\begin{equation}
    f = \frac{f'}{\gamma}
\end{equation}
$f'$ ist die ausgesandte Frequenz
\subsubsection{Effekte auf die Atomuhr im Satelliten}
Frequenz der Uhr ist $f_0=\SI{10.23}{\mega\hertz}$, alle internen Prozesse beruhen auf $f_0$
\begin{align}
    \delta^\text{rel} &= \frac{f_0'-f_0}{f_0} = \frac{1}{2}\left(\frac{v}{\symup{c}}\right)^{\!\!2}+\frac{\mu}{\symup{c}^2}\left[\frac{1}{R_\text{E}+h}-\frac{1}{r_\text{E}}\right] = \num{-4.464e-10}
    \intertext{Gravitationskonstante der Erde $\mu$, Erdradius $R_\text{E}=\SI{6370}{\kilo\meter}$, Satellitenhöhe $h=\SI{20200}{\kilo\meter}$}
    \symup{d}f &= \num{4.464e-10}f_0 = \SI{4.57e-3}{\hertz}
\end{align}
\subsubsection{Effekte auf die Empfängeruhr}
Geschwindigkeit des Empfängers
\begin{align}
    v &\approx \SI{0.5}{\kilo\meter\per\second}
    \shortintertext{damit folgt ein Shift von}
    \symup{d}f\propto\num{e-12}f_0
    \intertext{Nach 3 Stunden, $\SI{10}{\nano\second}$, Fehler von $\SI{30}{\centi\meter}$}
\end{align}
\begin{table}
    \centering
    \caption{Fehlerquellen und ihre Effekte}
    \begin{tabular}{c c}
        \toprule
        {Fehlerquelle} & {Effekt} \\
        \midrule
        Signal    & Reflektionen in Atmosphäre und an Gebäuden \\
        Satellit  & falsche Orbitale \\
                  & Systematische Uhrfehler \\
        Empfänger & $^{\---}"^{\---}$ \\
                  & variable Emfängerfrequenz \\
        \bottomrule
    \end{tabular}
\end{table}
