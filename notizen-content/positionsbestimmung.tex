\section{Positionsbestimmung}
\label{sec:positionsbestimmung}
\subsection{Korrekturen}
\label{sec:korrekturen}

\subsubsection{Zeitverzögerungen am Signal}
\label{sec:ZeitImSignal}
Erste Definitionen:
\begin{align}
    \text{S}   &= \text{Satellit}\\
    \text{E}   &= \text{Empfänger}\\
    t^\text{S} &= \text{Emissionszeit am Satelliten}\\
    δ^\text{S} &= \text{Verzögerungen im Satelliten, z.B. durch Rechenzeiten}\\
    t^\text{S} &= t^\text{S}(\text{GPS}) + δ^\text{S}\\
    t_\text{E} &= \text{Empfängerzeit}\\
    δ_\text{E} &= \text{Verzögerungen im Empfänger}\\
    t^\text{E} &= t^\text{E}(\text{GPS}) + δ^\text{E}
    \intertext{Damit folgt der Zeitunterschied}
    \increment t &= t_\text{E} - t^\text{S} = \increment t(\text{GPS}) +\incrementδ
    \intertext{und die Pseudoentfernung}
    R &= \symup{c}\increment t = ρ + \symup{c}\incrementδ\;.
    \intertext{Diese kann in eine Taylorreihe entwickelt werden}
    ρ\left(t_\text{E},\;t^\text{S}\right) &= ρ\left(t_\text{E},\;t^\text{S}+\increment t\right) = ρ\left(t^\text{S}\right)+\dot{ρ}\left(t^\text{S}\right)\cdot\increment t\;,
    \intertext{mit der radialen Geschwindigkeit  der Satelliten}
    \dot{ρ} &\approx \SI{0.9}{\kilo\meter\per\second}.
    \intertext{Die Laufzeit des Signal kann mit}
    \increment t &\approx\SI{0.07}{\second}
    \intertext{abgeschätzt werden, das führt zu einem Korrekturterm von}
    \dot{ρ}\left(t^\text{S}\right)\cdot\increment t &\approx\SI{60}{\meter}\;.
\end{align}

\subsubsection{Phasenverschiebung durch Doppler}
\begin{align}
    φ^\text{S}(t) &= \text{Phase des empfangenen Signals}\\
    φ_\text{E}(t) &= \text{Phase eines internen Referenzsignals}\\
    \intertext{Beide Signale starten bei}
    t_0 &= 0\;,
    \intertext{die beiden Phasen können dann mit}
    φ^\text{S}(t) &= f\cdot t-\frac{ρ}{\symup{c}}f+f\cdotδ^\text{S}\\
    φ_\text{E}(t) &= f\cdot t+f\cdotδ^\text{E}
    \intertext{beschrieben werden. Die relative Differenz der beiden Phasen ist}
    \frac{\symup{d}f}{f} &= \num{e-12}\;.
    \intertext{Bei einer Trägerfrequenz von}
    f &= \SI{1.5}{\giga\hertz}
    \intertext{folgt eine Ungenauigkeit von}
    \symup{d}f &= \SI{1.5e-3}{\hertz}\;.
    \intertext{Die absolute Differenz}
    φ^\text{S}_\text{E} &= -\frac{ρ}{\symup{c}}f-f\cdot\incrementδ
    \intertext{kann mit der Anzahl $N$ an empfangenen Signalen als}
    φ^\text{S}_\text{E} &= \incrementφ^\text{S}_\text{E}|^t_{t_0}+N
\end{align}
geschrieben werden.
Abbildung 6.1 aus {\small[H-W,L,C], S.90} zeigt dieses.
Die dicke Linie ist der Orbit und die dünne die ohne Phasewnverschiebung errechnete.

\subsubsection{Relativistische Effekte}
Von der Verscheibung der Frequenz
\begin{align}
    f &= f'\sqrt{1-\frac{v^2}{\symup{c}^2}}\\
    \frac{f_0'-f_0}{f_0} &= \num{-4.464e-10}
\end{align}
ist vor allem die Uhr betroffen.

\begin{table}
    \centering
    \caption{Fehlerquellen und ihre Effekte}
    \begin{tabular}{c c}
        \toprule
        {Fehlerquelle} & {Effekt} \\
        \midrule
        Signal    & Reflektionen in Atmosphäre und an Gebäuden \\
        Satellit  & falsche Orbitale \\
                  & Systematische Uhrfehler \\
        Empfänger & $^{\---}"^{\---}$ \\
                  & variable Emfängerfrequenz \\
        \bottomrule
    \end{tabular}
\end{table}


\subsection{Ortskoordinaten}
Wir folgen dem Unterkapitel \ref{sec:ZeitImSignal}.
\begin{align}
    R_\text{E}^\text{S}(t) &= ρ_\text{E}^\text{S}(t) + \symup{c}\incrementδ_\text{E}^\text{S}(t) \label{eqn:entfernung}\\
    ρ_\text{E}^\text{S}(t) &= \sqrt{\left(X^\text{S}(t)-X_\text{E}\right)^2+\left(Y^\text{S}(t)-Y_\text{E}\right)^2+\left(Z^\text{S}(t)-Z_\text{E}\right)^2}
    \intertext{Mit einem Satelliten kommt man so auf vier Unbekannte, die drei Koordinaten des Empfängers und die Uhrenfehler $δ$. Dieser wird wie folgt angenähert}
    δ(t) &= a_0 + a_1(t-t_\text{r})+ a_2(t-t_\text{r})^2\;,
    \intertext{mit einer Referenzzeit $t_\text{r}$. Damit kann die Gleichung \eqref{eqn:entfernung} so umgestellt werden, dass auf der linken Seite Messgrößen stehen und auf der rechten Seite die Unbekannten}
    R_\text{E}^\text{S}(t) + \symup{c}δ^\text{S}(t) &= ρ_\text{E}^\text{S}(t) + \symup{c}δ_\text{E}\;.
\end{align}
Zur Bestimmung der Position werden vier Satelliten benötigt, wenn die Höhe nicht interessiert, reichen drei Satelliten.

\begin{table}
    \centering
    \caption{Fehlerquellen und ihre Effekte}
    \begin{tabular}{c c}
        \toprule
        {Fehlerquelle} & {Effekt} \\
        \midrule
        Signal    & Reflektionen in Atmosphäre und an Gebäuden \\
        Satellit  & falsche Orbitale \\
                  & Systematische Uhrfehler \\
        Empfänger & $^{\---}"^{\---}$ \\
                  & variable Emfängerfrequenz \\
        \bottomrule
    \end{tabular}
\end{table}
