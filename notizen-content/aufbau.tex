\section{Aufbau}
\label{sec:aufbau}

\subsection{Stationen}
\begin{itemize}
    \item Bodenstationen, Synchronisierungsstelle ist Schriever Air Force Base in Colorado Springs
    \begin{enumerate}
        \item Monitor Station
        \begin{enumerate}
            \item aufzeichnen der Daten bei Überflug
            \item an Master Control Station weiterleiten
            \item 16 Stück, 6 Air Force, 10 National Geospital-Intelligence Agency
        \end{enumerate}
        \item Master Control Station
        \begin{enumerate}
            \item Berechnen der Präzession der Satelliten
            \item Navigationsanweisungen berechnen
            \item Sicherstellung der Langlebigkeit und Genauigkeit
            \item Positionierung der Satelliten für eine bestmögliche Konstellation
        \end{enumerate}
        \item Boden-Antennen:
        \begin{enumerate}
            \item Senden Befehle, Software-Updates an Satelliten
            \item 4 GPS eigene, 7 \textbf{A}ir \textbf{F}orce \textbf{S}atellite \textbf{C}ontrol \textbf{N}etwork
        \end{enumerate}
    \end{enumerate}
    \item Konstellation
    \begin{enumerate}
        \item 20 200km Höhe
        \item 2 Umläufe pro Tag pro Satellit
        \item 6 gleichmäßig verteilte Orbitale mit 4 Satelliten pro Bahn -> 24 Satelliten, an jedem Punkt min 4 Satelliten erreichbar
        \item Ersatzsatelliten falls längere Wartungsarbeiten anstehen, Ausfälle passieren
        \item 2011: Umpositionierung, \textit{Expandable 24}, jetzt 27, bessere Abdeckung in vielen Teilen der Erde
    \end{enumerate}
\end{itemize}

\subsection{Satelliten}
\begin{table}
    \caption{Die verschiedenen Satelliten im Überblick.}
    \label{tab:satelliten}
    \begin{tabular}{c S[table-format=2.0] c S[table-format=2.1] c}
        \toprule
        {Generation} & {in Nutzung} & {Start} & {Lebensspanne\;/\;$a$} & {Änderungen} \\
        \midrule
        Block IIA   &  1 & 1990-1997 & 7.5 & \\
        Block IIR   & 11 & 1997-2004 & 7.5 & Onboard Uhrüberwachung \\
        Block IIR-M &  7 & 2005-2009 & 7.5 & Mehr und stärkere M(ilitär)signale \\
        Block IIF   & 12 & 2010-2016 & 12  & neue Uhr, bessere Signalabstrahlung \\
        GPS III     &  0 & Dez 2018  & 15  & bessere Zuverlässigkeit, kein SA mehr \\
        GPS IIIF    &  0 & Dez 2018  & 15  & " , SAR kompatibel \\
        \bottomrule
    \end{tabular}
\end{table}
SA = Selective Availability; Möglichkeit das zivile GPS zu stören, falsche Zeiten einzuspeisen, \textit{national security reasons}, 02.05.2000, 4 Uhr abgeschaltet,
Erste Erwartung: auf $\SI{400}{\meter}$ genau, Tests zeigen $15 - \SI{40}{\meter}$, $\increment v\leg \SI{1}{\meter\per\second}$
