\section{Signale}
\label{sec:Signale}
Grundfrequenzen
\begin{align}
    f_0 &= \SI{10.23}{\mega\hertz}  &\text{fundamentale Frequenz}\\
    L1 &= \SI{1575.42}{\mega\hertz} &\text{erste Trägerfrequenz}\\
    L2 &= \SI{1227.60}{\mega\hertz} &\text{zweite Trägerfrequenz}
\end{align}
Durch Multiplikation mit 154 und 120\\
PRN = pseudorandom noise auf Frequenzen\\
+ Datensignal\\~\\
Kodierungen:\\
C/A = Coarse/Acquisition-codeS (Zivil), für das \textbf{S}tandard \textbf{P}ositioning \textbf{S}ystem, $\SI{300}{\meter}$ Wellenlänge, $f=\SI{1.023}{\mega\hertz}$\\
P-code = \textbf{P}recision-Code, Militär, $f=\SI{10.23}{\mega\hertz}$\\
C = Civil

\begin{table}
    \centering
    \caption{Frequenzverteilung auf den verschiedenen Generationen von Satelliten.}
    \label{tab:frequenzen}
    \begin{tabular}{c c c c c c}
        \toprule
        {Satellit} & {L1 C/A} & {L1 P(Y)} & {L2C} & {L5} & {L1C} \\
        & {$\SI{1575.42}{\mega\hertz}$} & {$\SI{1575.42}{\mega\hertz}$} & {$\SI{1227.60}{\mega\hertz}$} & {$\SI{1176.45}{\mega\hertz}$} & {$\SI{1575.42}{\mega\hertz}$} \\
        \midrule
        Block IIA   & \ch &     &     &     &     \\
        Block IIR   & \ch & \ch &     &     &     \\
        Block IIR-M & \ch & \ch & \ch &     &     \\
        Block IIF   & \ch & \ch & \ch & \ch &     \\
        GPS III     & \ch & \ch & \ch & \ch & \ch \\
        GPS IIIF    & \ch & \ch & \ch & \ch & \ch \\
        \bottomrule
    \end{tabular}
\end{table}

\begin{multicols}{2}
    Informationen im Signal
    \begin{itemize}
        \item Satellitenposition
        \item Zeit
        \item Uhrzeitkorrekturen
        \item Systeminformationen
    \end{itemize}
    \columnbreak
    Wellenkomponenten
    \begin{enumerate}
        \item Sinusträger
        \item Binärdaten
    \end{enumerate}
\end{multicols}
