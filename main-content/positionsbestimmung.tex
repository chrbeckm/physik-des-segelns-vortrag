\section{Positionsbestimmung}

\begin{frame}{Phasenverschiebung}
    \begin{columns}
        \begin{column}{0.5\textwidth}
            \begin{align}
                φ^\text{S}(t) &= f\cdot t-\frac{ρ}{\symup{c}}f+f\cdotδ^\text{S} \\
                φ_\text{E}(t) &= f\cdot t+f\cdotδ^\text{E} \\
                φ^\text{S}_\text{E} &= -\frac{ρ}{\symup{c}}f-f\cdot\incrementδ \\
                &= \incrementφ^\text{S}_\text{E}|^t_{t_0}+N \\
                N &= \text{\# empfangene Signale}
            \end{align}
        \end{column}
        \begin{column}{0.5\textwidth}
            \begin{figure}
                \includegraphics[width=\textwidth]{images/phasenverschiebung.jpg}
            \end{figure}
            \centering{\small[C,H-W,L]}
        \end{column}
    \end{columns}
\end{frame}

\begin{frame}{Zeitverzögerungen am Signal}
    S = Satellit, E = Empfänger
    \begin{align}
        t &= \text{Sende- /Empfangszeit} \\
        δ &= \text{zeitliche Verzögerungen, z.B. durch Rechenzeiten} \\
        t &= t(\text{GPS}) + δ
        \shortintertext{Zeitunterschied:}
        \increment t &= t_\text{E} - t^\text{S} = \increment t(\text{GPS}) +\incrementδ
        \shortintertext{Pseudoentfernung:}
        R &= \symup{c}\increment t = ρ + \symup{c}\incrementδ
        \shortintertext{Taylorreihe:}
        ρ\left(t_\text{E},\;t^\text{S}\right) &= ρ\left(t_\text{E},\;t^\text{S}+\increment t\right) = ρ\left(t^\text{S}\right)+\dot{ρ}\left(t^\text{S}\right)\cdot\increment t
        \shortintertext{Größe der ersten Korrektur:}
        \dot{ρ}\left(t^\text{S}\right)\cdot\increment t &\approx \SI{0.9}{\kilo\meter\per\second}\cdot\SI{0.07}{\second} =\SI{63}{\meter}
    \end{align}
\end{frame}

\begin{frame}{Ortskoordinaten}

\end{frame}
